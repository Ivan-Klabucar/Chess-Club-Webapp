\chapter{Zaključak i budući rad}
		
		Zadatak grupe MISFLIP bila je izrada web aplikacije koja omogućuje poslovanje šahovskom klubu s naglašenom mogućnošću rješavanja dnevnih taktika kako neregistriranih korisnika tako i članova, te prijava članova na treninge i turnire. Grupa od sedam članova zadatak je izvršila uspješno kroz tri faze u zadanom vremenskom roku od 17 tjedana.
		
		Prva faza uključivala je formiranje tima i uvodni sastanak putem MS Teamsa s asistentom gdje smo dobili uvodne upute i savjete vezane uz izvršavanje projekta.
		
		Druga faza odnosila se na period gdje smo podijelili zaduženja prema afinitetima a sve probleme i tijek izrade dodatno komentirali na tjednim sastancima članova preko MS Teamsa. Dio komunikacije odvijao se i preko Whatsappa. Glavnina ove faze bila je vezana uz izradu dokumentacije, a pri samom kraju uz navedeno počelo se raditi na programskom rješenju.
		
		U trećoj fazi intenzitet rada i komunikacije znatno je povećan. Paralelno se radilo na dokumentaciji i programiranju. U ovoj fazi grupa je imala određenih poteškoća zbog dodatnih napora učenja i savladavanja i gradiva predmeta i programske tehnologije.
		
		Grupa je uložila velik trud i napor, ali i ostavila prostor za proširenje i unapređenje. Sama provedba koliko god bila zahtjevna, ujedinila je grupu u jednu cjelinu i pokazala kako kolektiv savladava zadani im projekt. Naučili smo puno, ali i imali značajnu podršku asistenta.
		
		
		\eject 