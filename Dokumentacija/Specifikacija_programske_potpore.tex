\chapter{Specifikacija programske potpore}
		
	\section{Funkcionalni zahtjevi}
			
			\textbf{\textit{dio 1. revizije}}\\
			
			\textit{Navesti \textbf{dionike} koji imaju \textbf{interes u ovom sustavu} ili  \textbf{su nositelji odgovornosti}. To su prije svega korisnici, ali i administratori sustava, naručitelji, razvojni tim.}\\
				
			\textit{Navesti \textbf{aktore} koji izravno \textbf{koriste} ili \textbf{komuniciraju sa sustavom}. Oni mogu imati inicijatorsku ulogu, tj. započinju određene procese u sustavu ili samo sudioničku ulogu, tj. obavljaju određeni posao. Za svakog aktora navesti funkcionalne zahtjeve koji se na njega odnose.}\\
			
			
			\noindent \textbf{Dionici:}
			
			\begin{packed_enum}
				
				\item Vlasnik (naručitelj)				
				\item Zaposlenici šahovskog kluba
				\begin{packed_enum}
						
					\item Treneri
						
				\end{packed_enum}
				\item Članovi šahovskog kluba
				\item Neregistrirani korisnici aplikacije
				\item Administrator
				\item Razvojni tim
				
			\end{packed_enum}
			
			\noindent \textbf{Aktori i njihovi funkcionalni zahtjevi:}
			
			
			\begin{packed_enum}
				\item  \underbar{Trener (inicijator) može:}
				
				\begin{packed_enum}
					
					\item postavljati dnevne šahovske taktike
					\item slagati raspored vlastitih treninga
					\item organizirati turnire
					\item vidjeti rang liste članova
					\item pristupiti i objavljivati novi sadržaj na stranici novosti 
					
				\end{packed_enum}
			
				\item  \underbar{Član šahovskog kluba (inicijator) može:}
				
				\begin{packed_enum}
					
					\item platiti članarinu putem aplikacije
					\item prijavljivati se za treninge kod pojedinih trenera
					\item prijavljivati se za turnire
					\item rješavati dnevne šahovske taktike
					\begin{packed_enum}
						
						\item nakon rješavanja dnevne šahovske taktike mogu joj dodijeliti ocjenu
						\item nakon rješavanja dnevne šahovske taktike mogu prijaviti grešku u taktici
						
					\end{packed_enum}
					\item vidjeti rang liste članova
					\item pristupiti stranici novosti
					
				\end{packed_enum}
			
				\item \underbar{Administrator (inicijator) može:}
				
				\begin{packed_enum}
					
					\item potpuno zabraniti pristup bilo kojem članu ili treneru
					\item zabraniti pristup bilo čemu \textbf{osim} uplate članarine bilo kojem članu
					\item vidjeti rang liste članova
					\item objavljivati i skidati sadržaj na stranici novosti
					\item mijenjati raspored treninga bilo kojem treneru
					\item postavljati i skidati dnevne šahovske taktike
					\begin{packed_enum}
						
						\item nakon rješavanja dnevne šahovske taktike mogu joj dodijeliti ocjenu
						\item nakon rješavanja dnevne šahovske taktike mogu prijaviti grešku u taktici
						
					\end{packed_enum}
					\item dodavati i skidati turnire
					\item pregledavati transakcije
					
				\end{packed_enum}
			
				\item \underbar{Neregistrirani korisnik (inicijator) može:}
				
				\begin{packed_enum}
					
					\item rješavati dnevne šahovske taktike
					\item vidjeti rang liste članova
					\item pristupiti novostima
					
				\end{packed_enum}
			
				\item \underbar{Baza podataka (sudionik):}
				
				\begin{packed_enum}
					
					\item pohranjuje sve podatke o korisnicima i njihovim ovlastima
					\item pohranjuje sve dnevne šahovske taktike
					\item pohranjuje rang listu članova
					\item pohranjuje povijest svih transakcija
					\item pohranjuje termine svih treninga
					\item pohranjuje termine svih turnira
					\item pohranjuje svaku stavku na stranici novosti
					
				\end{packed_enum}
				
			\end{packed_enum}
			
			\eject 
			
			
				
			\subsection{Obrasci uporabe}
				
				\textbf{\textit{dio 1. revizije}}
				
				\subsubsection{Opis obrazaca uporabe}
					\textit{Funkcionalne zahtjeve razraditi u obliku obrazaca uporabe. Svaki obrazac je potrebno razraditi prema donjem predlošku. Ukoliko u nekom koraku može doći do odstupanja, potrebno je to odstupanje opisati i po mogućnosti ponuditi rješenje kojim bi se tijek obrasca vratio na osnovni tijek.}\\
					

					\noindent \underbar{\textbf{UC$<$broj obrasca$>$ -$<$ime obrasca$>$}}
					\begin{packed_item}
	
						\item \textbf{Glavni sudionik: }$<$sudionik$>$
						\item  \textbf{Cilj:} $<$cilj$>$
						\item  \textbf{Sudionici:} $<$sudionici$>$
						\item  \textbf{Preduvjet:} $<$preduvjet$>$
						\item  \textbf{Opis osnovnog tijeka:}
						
						\item[] \begin{packed_enum}
	
							\item $<$opis korak jedan$>$
							\item $<$opis korak dva$>$
							\item $<$opis korak tri$>$
							\item $<$opis korak četiri$>$
							\item $<$opis korak pet$>$
						\end{packed_enum}
						
						\item  \textbf{Opis mogućih odstupanja:}
						
						\item[] \begin{packed_item}
	
							\item[2.a] $<$opis mogućeg scenarija odstupanja u koraku 2$>$
							\item[] \begin{packed_enum}
								
								\item $<$opis rješenja mogućeg scenarija korak 1$>$
								\item $<$opis rješenja mogućeg scenarija korak 2$>$
								
							\end{packed_enum}
							\item[2.b] $<$opis mogućeg scenarija odstupanja u koraku 2$>$
							\item[3.a] $<$opis mogućeg scenarija odstupanja  u koraku 3$>$
							
						\end{packed_item}
					\end{packed_item}
				
					
				\subsubsection{Dijagrami obrazaca uporabe}
					
					\textit{Prikazati odnos aktora i obrazaca uporabe odgovarajućim UML dijagramom. Nije nužno nacrtati sve na jednom dijagramu. Modelirati po razinama apstrakcije i skupovima srodnih funkcionalnosti.}
				\eject		
				
			\subsection{Sekvencijski dijagrami}
				
				\textbf{\textit{dio 1. revizije}}\\
				
				\textit{Nacrtati sekvencijske dijagrame koji modeliraju najvažnije dijelove sustava (max. 4 dijagrama). Ukoliko postoji nedoumica oko odabira, razjasniti s asistentom. Uz svaki dijagram napisati detaljni opis dijagrama.}
				\eject
	
		\section{Ostali zahtjevi}
		
			\textbf{\textit{dio 1. revizije}}\\
		 
			 \textit{Nefunkcionalni zahtjevi i zahtjevi domene primjene dopunjuju funkcionalne zahtjeve. Oni opisuju \textbf{kako se sustav treba ponašati} i koja \textbf{ograničenja} treba poštivati (performanse, korisničko iskustvo, pouzdanost, standardi kvalitete, sigurnost...). Primjeri takvih zahtjeva u Vašem projektu mogu biti: podržani jezici korisničkog sučelja, vrijeme odziva, najveći mogući podržani broj korisnika, podržane web/mobilne platforme, razina zaštite (protokoli komunikacije, kriptiranje...)... Svaki takav zahtjev potrebno je navesti u jednoj ili dvije rečenice.}
			 
			 
			 
	