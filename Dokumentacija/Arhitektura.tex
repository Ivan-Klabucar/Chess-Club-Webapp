\chapter{Arhitektura i dizajn sustava}
		 Izbor odgovarajuće arhitekture programske poruke jedan je od najbitnijih koraka u oblikovanju sustava jer ona predstavlja poveznicu između zahtjeva na sustav i same implementacije sustava. Dobra arhitektura povlači dobru fleksibilnost sustava, jednostavnu mogućnost nadogradnje i jeftino održavanje.
		

\noindent S obzirom na to da je zadatak ovog projekta napraviti aplikaciju za šahovski klub, logičan izbor je web aplikacija. Primarni razlog je to što web aplikacija ne ovisi o platformi, već radi na svakom sustavu koji ima web preglednik što znatno smanjuje vrijeme i troškove potrebne za razvoj za više platformi
		

\noindent Radni okvir koji smo odabrali za arhitekturu je Django (pisan u i za jezik Python). S njime smo arhitekturu sustava odlučili temeljiti na MVT (Model-View-Template) konceptu kojeg on nativno podržava. Jedina bitna razlika između njega i poznatog MVC (Model-View-Controller) koncepta je to što u Django sam po sebi sređuje "controller" dio (programski kod koji kontrolira interakciju između modela i viewa) i daje nam na raspolaganje Template.
		
		Dakle aplikacija se sastoji od 3 dijela:
		\begin{itemize}
			\item 	Model - Sređuje apstrakciju podataka, služi kao sučelje za podatke spremljene u bazu i dopušta upravljanje njima bez prevelikog razumijevanja kompleksnosti same baze 
			\item 	View - funkcionalnosti slične controlleru u MVC konceptu. Sređuje svu logiku koja se treba prikazati na templateu i služi kao "most" između modela i templatea
			\item 	Template - funkcionalnosti slične Viewu u MVC konceptu. Služi kao sloj prikaza sadržaja i zadužen je za to kako i što će biti prikazano korisniku. Specifično za Django, Template je vrsta HTML datoteke koja može koristiti Django Template Language (DTL) što nam znatno olakšava komunikaciju između frontenda i backenda aplikacije.	
		\end{itemize}
		

				
		\section{Baza podataka}
		
		Za sustav baza podataka za ovaj projekt smo odabrali PostgreSQL. 
		
		PostgreSQL je objektno-relacijski sustav upravljanja bazama podataka (ORDBMS) zasnovan na POSTGRES-u, inačica 4.2, razvijen na Kalifornijskom sveučilištu na Odjelu za računalne znanosti Berkeley. POSTGRES je pionir mnogih koncepata koji su tek kasnije postali dostupni u nekim komercijalnim sustavima baza podataka.
		
		PostgreSQL je "open-source" potomak ovog izvornog Berkeley koda. Podržava velik dio SQL standarda i nudi brojne moderne značajke poput:
		\begin{packed_item}
			\item složenih upita,
			\item stranih ključeva,
			\item okidača,
			\item obnovljivih pogleda,
			\item transakcijski integritet te
			\item multiverzijski protokol kontrole istodobnog
			pristupa
		\end{packed_item}
		
		Također, korisnik može proširiti PostgreSQL na mnogo načina, na primjer dodavanjem novih:
		\begin{packed_item}
			\item vrsta podataka,
			\item funkcija,
			\item operatora,
			\item agregatnih funkcija,
			\item indeksnih metoda te
			\item proceduralnih jezika.
		\end{packed_item}
		
		
		A zbog liberalne licence, PostgreSQL može koristiti, mijenjati i distribuirati bilo tko u bilo koju svrhu, bilo privatnu, komercijalnu ili akademsku.\footnote{\url{https://www.postgresql.org/docs/current/intro-whatis.html}}
		
		U relacijskim bazama podataka, osnovne gradivne jedinice su tablice. Tablice koje grade bazu podataka ovog projekta su:
		\begin{packed_item}
			\item taktika
			\item rjesenjeTaktika
			\item dojavaPogreske
			\item view - rangLista
			\item Korisnik
			\item Grupa
			\item Novost
			\item Trening
			\item turnir
			\item PrijavaTrening
			\item PrijavaTurnir
			\item Transakcija
			
		\end{packed_item}
		
		\eject
		
			\subsection{Opis tablica}
			Primarni ključevi tablice označeni su sa slovom 'P', a strani ključevi sa slovom 'F'.\\
			
			\noindent Entitet \textbf{Taktika} u odnosu je \textit{One to Many} s entitetima \textbf{Rješenje taktika} i \textbf{Dojava pogreške}.
				\begin{longtabu} to \textwidth {|X[6, l]|X[6, l]|X[20, l]|}
					
					\hline \multicolumn{3}{|c|}{\textbf{taktika}}	 \\[3pt] \hline
					\endfirsthead
					
					\hline \multicolumn{3}{|c|}{\textbf{taktika}}	 \\[3pt] \hline
					\endhead
					
					\hline 
					\endlastfoot
					
					\cellcolor{LightGreen}idTaktika P& INT	   &  identifikacijski broj taktike	\\ \hline
					createdAt	& TIMESTAMP &   vrijeme stvaranja taktike, služi za kronološko sortiranje taktika	\\ \hline 
					\cellcolor{LightBlue}idUser F& INT & identifikator trenera ili admina koji je stvorio taktiku  \\ \hline 
					initConfig & VARCHAR (100)	&  string koji opisuje inicijalno stanje šahovske ploče	\\ \hline 
					movesWhite & VARCHAR (3000) & string koji opisuje poteze bijelih figurica \\ \hline
					movesBlack & VARCHAR (3000) & string koji opisuje poteze crnih figurica \\ \hline
					tezina & DECIMAL & težina taktike, koristi se u računanju bodova, u rasponu od 1 do 3 \\ \hline
					brojGlasova & INT & broj korisnika koji su glasali za neku težinu taktike \\ \hline

				\end{longtabu}
				
				\noindent Entitet \textbf{Rješenje taktika} u odnosu je \textit{One to One} s entitetnom \textbf{Taktika}, te u odnosu  \textit{One to One} s entitetnom \textbf{Korisnik}.
				\begin{longtabu} to \textwidth {|X[6, l]|X[6, l]|X[20, l]|}
					
					\hline \multicolumn{3}{|c|}{\textbf{rjesenjeTaktika}}	 \\[3pt] \hline
					\endfirsthead
					
					\hline \multicolumn{3}{|c|}{\textbf{rjesenjeTaktika}}	 \\[3pt] \hline
					\endhead
					
					\hline 
					\endlastfoot
					
					\cellcolor{LightGreen}idTaktika PF& INT	   &  identifikacijski broj taktike koja je rješena, foreign key	\\ \hline
					\cellcolor{LightGreen}idUser PF& INT & identifikator člana koji je riješio taktiku, foreign key  \\ \hline 
					vrijeme & DECIMAL & vrijeme rješavanja taktike u minutama \\ \hline
					
				\end{longtabu}
				\eject
				
				\noindent Entitet \textbf{Dojava pogreške} u odnosu je \textit{One to One} s entitetnom \textbf{Taktika}, te u odnosu  \textit{One to One} s entitetnom \textbf{Korisnik} koji je prijavio taktiku. Također je u odnosu \textit{One to One} s entitetnom \textbf{Korisnik} koji predstavlja trenera koji je zadužen za revidiranje dojave.
				\begin{longtabu} to \textwidth {|X[6, l]|X[6, l]|X[20, l]|}
					
					\hline \multicolumn{3}{|c|}{\textbf{dojavaPogreske}}	 \\[3pt] \hline
					\endfirsthead
					
					\hline \multicolumn{3}{|c|}{\textbf{dojavaPogreske}}	 \\[3pt] \hline
					\endhead
					
					\hline 
					\endlastfoot
					
					\cellcolor{LightGreen}idDojava P& INT	   &  identifikacijski broj dojave o pogrešci	\\ \hline
					\cellcolor{LightBlue}idTaktika F& INT	   &  identifikacijski broj prijavljene taktike	\\ \hline
					\cellcolor{LightBlue}idUserDojavio F& INT & identifikator člana koji je dojavio pogrešku na taktiku  \\ \hline 
					\cellcolor{LightBlue}idUserRevidira F& INT & identifikator trenera koji revidira dojavu  \\ \hline 
					prihvacena & BOOLEAN	&  je li dojava prihvaćena ili odbačena, NULL znači da čeka na revidiranje	\\ \hline 
					predlozeniTijek & VARCHAR (6000) & string koji opisuje poteze koje je predložio član koji je dojavio pogrešku \\ \hline
					opisPoteza & TEXT & opis poteza koje je član unio \\ \hline
					
				\end{longtabu}
				
				\noindent Entiteti u tablici \textbf{rangLista} u odnosu su \textit{One to One} s entitetnom \textbf{Korisnik}.
				\begin{longtabu} to \textwidth {|X[6, l]|X[6, l]|X[20, l]|}
					
					\hline \multicolumn{3}{|c|}{\textbf{view - rangLista}}	 \\[3pt] \hline
					\endfirsthead
					
					\hline \multicolumn{3}{|c|}{\textbf{view - rangLista}}	 \\[3pt] \hline
					\endhead
					
					\hline 
					\endlastfoot
					
					\cellcolor{LightGreen}idUser PF& INT	   &  identifikacijski broj člana	\\ \hline
					username & VARCHAR	   &  korisničko ime člana	\\ \hline
					bodovi & INT & broj bodova koje je član sveukupno osvojio \\ \hline 
					
				\end{longtabu}
			
				\noindent Entitet \textbf{Korisnik} u odnosu je \textit{One to Many} s entitetima \textbf{Rješenje taktika, Grupa} i \textbf{Dojava pogreške}, te u odnosu \textit{One to One} s entitetima iz tablice \textbf{rangLista}.
				\begin{longtabu} to \textwidth {|X[6, l]|X[8, l]|X[20, l]|}
					
					\hline \multicolumn{3}{|c|}{\textbf{Korisnik}}	 \\[3pt] \hline
					\endfirsthead
					
					\hline \multicolumn{3}{|c|}{\textbf{Korisnik}}	 \\[3pt] \hline
					\endhead
					
					\hline 
					\endlastfoot
					
					\cellcolor{LightGreen} idUser P& INT &  identifikacijski broj člana \\ \hline
					\cellcolor{LightBlue} idGrupa F& INT & identifikacijski broj grupe \\ \hline
					username & VARCHAR (20) &  korisničko ime člana \\ \hline 
					password & VARCHAR (20) & lozinka člana \\ \hline 
					email & VARCHAR (20) & mail adresa člana \\ \hline
					ime & VARCHAR (20) & ime člana \\ \hline
					prezime & VARCHAR (20) & prezime člana \\ \hline
					
				\end{longtabu}
			
				\begin{longtabu} to \textwidth {|X[6, l]|X[8, l]|X[20, l]|}
				
					\hline \multicolumn{3}{|c|}{\textbf{Grupa}}	 \\[3pt] \hline
					\endfirsthead
					
					\hline \multicolumn{3}{|c|}{\textbf{Grupa}}	 \\[3pt] \hline
					\endhead
					
					\hline 
					\endlastfoot
					
					\cellcolor{LightGreen} idGrupa P& INT & identifikacijski broj grupe \\ \hline
					imeGrupe & VARCHAR (20) &  ime grupe \\ \hline
				
				\end{longtabu}
				
				\noindent Entitet \textbf{Novost} u odnosu je \textit{Many to One} s entitetom \textbf{Korisnik} 
				\begin{longtabu} to \textwidth {|X[10, l]|X[8, l]|X[20, l]|}
				
					\hline \multicolumn{3}{|c|}{\textbf{Novost}}	 \\[3pt] \hline
					\endfirsthead
					
					\hline \multicolumn{3}{|c|}{\textbf{Novost}}	 \\[3pt] \hline
					\endhead
					
					\hline 
					\endlastfoot
					
					\cellcolor{LightGreen} idUser PF& INT & identifikacijski broj autora \\ \hline
					\cellcolor{LightGreen} vrijemeObjavljivanja P& DATETIME & vrijeme i datum objavljivanja \\ \hline
					naslov & VARCHAR (50) & naslov novosti\\ \hline
					tekst & TEXT & tekst novosti\\ \hline
				
				\end{longtabu}

				\noindent Entitet \textbf{Trening} u odnosu je \textit{One to Many} s entitetom \textbf{PrijavaTrening}, te u odnosu \textit{Many to One}  s entitetom \textbf{Korisnik}.

				\begin{longtabu} to \textwidth {|X[10, l]|X[8, l]|X[20, l]|}
				
					\hline \multicolumn{3}{|c|}{\textbf{Trening}}	 \\[3pt] \hline
					\endfirsthead
					
					\hline \multicolumn{3}{|c|}{\textbf{Trening}}	 \\[3pt] \hline
					\endhead
					
					\hline 
					\endlastfoot
					
					\cellcolor{LightGreen} idTreninga P& INT & identifikacijski broj treninga\\ \hline
					\cellcolor{LightBlue} idOrganizatora F& INT & identifikacijski broj organizatora treninga \\ \hline
					vrijemePocetka & DATETIME & vrijeme i datum početka treninga\\ \hline
					vrijemeZavrsetka & DATETIME & vrijeme i datum završetka treninga\\ \hline
					opisTreninga & TEXT & opisan sadržaj treninga\\ \hline
				
				\end{longtabu}
				
				\noindent Entitet \textbf{Turnir} u odnosu je \textit{One to Many} s entitetom \textbf{PrijavaTurnir}.

				\begin{longtabu} to \textwidth {|X[10, l]|X[8, l]|X[20, l]|}
				
					\hline \multicolumn{3}{|c|}{\textbf{Turnir}}	 \\[3pt] \hline
					\endfirsthead
					
					\hline \multicolumn{3}{|c|}{\textbf{Turnir}}	 \\[3pt] \hline
					\endhead
					
					\hline 
					\endlastfoot
					
					\cellcolor{LightGreen} idTurnira P& INT & identifikacijski broj turnira\\ \hline
					formatTurnira & TEXT & opisan format turnira \\ \hline
					vrijemePocetka & DATETIME & vrijeme i datum početka turnira\\ \hline
					vrijemeZavrsetka & DATETIME & vrijeme i datum završetka turnira\\ \hline
					brojSudionika & INT & maksimalan broj sudionika na turniru\\ \hline
				
				\end{longtabu}

				\noindent Entitet \textbf{PrijavaTrening} u odnosu je \textit{Many to One} s entitetom \textbf{Trening}, te u odnosu \textit{One to One}  s entitetom \textbf{Korisnik}.

				\begin{longtabu} to \textwidth {|X[10, l]|X[8, l]|X[20, l]|}
				
					\hline \multicolumn{3}{|c|}{\textbf{PrijavaTrening}}	 \\[3pt] \hline
					\endfirsthead
					
					\hline \multicolumn{3}{|c|}{\textbf{PrijavaTrening}}	 \\[3pt] \hline
					\endhead
					
					\hline 
					\endlastfoot
					
					\cellcolor{LightBlue} idClana PF& INT & identifikacijski broj člana\\ \hline
					\cellcolor{LightBlue} idTreninga PF & INT & identifikacijski broj treninga \\ \hline
				
				\end{longtabu}

				\noindent Entitet \textbf{PrijavaTurnir} u odnosu je \textit{Many to One} s entitetom \textbf{Turnir}, te u odnosu \textit{One to One}  s entitetom \textbf{Korisnik}.

				\begin{longtabu} to \textwidth {|X[10, l]|X[8, l]|X[20, l]|}
				
					\hline \multicolumn{3}{|c|}{\textbf{PrijavaTurnir}}	 \\[3pt] \hline
					\endfirsthead
					
					\hline \multicolumn{3}{|c|}{\textbf{PrijavaTurnir}}	 \\[3pt] \hline
					\endhead
					
					\hline 
					\endlastfoot
					
					\cellcolor{LightBlue} idClana PF& INT & identifikacijski broj člana\\ \hline
					\cellcolor{LightBlue} idTurnira PF& INT & identifikacijski broj turnira \\ \hline
				
				\end{longtabu}
	
				\noindent Entitet \textbf{Transakcija} u odnosu je \textit{Many to One} s entitetom \textbf{Korisnik}.

				\begin{longtabu} to \textwidth {|X[10, l]|X[8, l]|X[20, l]|}
				
					\hline \multicolumn{3}{|c|}{\textbf{Transakcija}}	 \\[3pt] \hline
					\endfirsthead
					
					\hline \multicolumn{3}{|c|}{\textbf{Transakcija}}	 \\[3pt] \hline
					\endhead
					
					\hline 
					\endlastfoot
					
					\cellcolor{LightGreen} idTransakcije P& INT & identifikacijski broj transakcije\\ \hline
					\cellcolor{LightBlue} idClana F& INT & identifikacijski broj člana\\ \hline
					datumTransakcije & DATETIME & vrijeme i datum izvršavanja transakcije\\ \hline
					iznosUplate & DECIMAL & uplaćen iznos\\ \hline

				\end{longtabu}

				
			
			\subsection{Dijagram baze podataka}
			
			\begin{figure}[H]
					\centerfloat
        					\includegraphics[scale=0.60]{dijagrami/dijagramBaze.png} %veličina slike u odnosu na originalnu datoteku i pozicija slike
        					\caption{E-R dijagram baze podataka}
        					\label{fig:DBdiagram}
				\end{figure}
			
			\eject
			
			
		\section{Dijagram razreda}

			\text{Dijagram modela potrebnih za funkcionalnosti autentikacije.}
			\begin{figure}[H]
					\centerfloat
					\advance\leftskip0.7cm
        					\includegraphics[scale=0.45]{dijagrami/ModelsClassDiagram1.png} %veličina slike u odnosu na originalnu datoteku i pozicija slike
        					\caption{Dijagram razreda modela generične funkcionalnosti}
        					\label{fig:DijagramRazredaModel}
			\end{figure}
			
			\begin{figure}[H]
					\centerfloat
					\advance\leftskip0.7cm
        					\includegraphics[scale=0.65]{dijagrami/ViewClassDiagram1.png} %veličina slike u odnosu na originalnu datoteku i pozicija slike
        					\caption{Dijagram razreda view-ova generične funkcionalnosti}
        					\label{fig:DijagramRazredaView}
			\end{figure}
			
			\eject
			
			\begin{figure}[H]
					\centerfloat
					\advance\leftskip0.7cm
        					\includegraphics[scale=0.35]{dijagrami/Classdiagram.jpg} %veličina slike u odnosu na originalnu datoteku i pozicija slike
        					\caption{Idejni dijagram razreda}
        					\label{fig:idejniDijagramRazreda}
				\end{figure}

			
			\textbf{\textit{dio 2. revizije}}\\			
			
			\textit{Prilikom druge predaje projekta dijagram razreda i opisi moraju odgovarati stvarnom stanju implementacije}
			
			
			
			\eject
		
		\section{Dijagram stanja}
			
			
				Na slici 4.5 prikazan je dijagram stanja za neregistriranog korisnika. Pri otvaranju aplikacije, korisniku se otvara početna stranica (Novosti) na kojoj može pročitati sve novosti koje su objavljene. Klikom na Login korisniku se otvara stranica na kojoj se može prijaviti upisivanjem korisničkog imena (Username) i lozinke (Password), no pošto nije registriran neće se moći prijaviti i koristiti dodatne mogućnosti aplikacije. Korisniku se klikom na „Register“ otvara stranica na kojoj može stvoriti korisnički račun unošenjem korisničkog imena (Username) i lozinke (Password). Ako korisnik zadovoljava sve uvjete vezane uz odabir korisničkog imena i lozinke pritiskom na Sign up korisnik stvara korisnički račun.
				Također, klikom na „Dnevne taktike“ korisniku se otvara stranica na kojoj može odabrati dnevnu taktiku i započeti s njezinim rješavanjem. Rješavajući taktiku, u slučaju odabira krivog poteza, korisniku se javlja poruka „Krivi potez!“ te pritiskom na „Zatvori“ korisnik može nastaviti s rješavanjem dnevne taktike. U slučaju odabira posljednjeg traženog točnog poteza, korisniku se javlja poruka „Uspješno ste riješili dnevnu taktiku!“ i klikom na „Povratak na Dnevne taktike“ korisnik se vraća na stranicu „Dnevne taktike“ gdje može odabrati već riješenu ili neku drugu taktiku.
				
			
			
			\begin{figure}[H]
				\centerfloat
				\includegraphics[scale=0.21]{dijagrami/Dijagram stanja - Neregistriran korisnik.jpg} %veličina slike u odnosu na originalnu datoteku i pozicija slike
				\caption{Dijagram stanja za neregistriranog korisnika}
				\label{fig:UC9}
			\end{figure}
			
			\eject
			
			
		\section{Dijagram aktivnosti}
		
		Na slici 4.6 prikazan je proces revidiranja pogreške u taktici. Trener otvara pregled s detaljima o dojavi pogreške u taktici i u ovisnosti o tome je li dojava o pogrešci valjana potvrđuje ili odbija dojavu. U slučaju da prihvati dojavu, taktika se automatski revidira te se rang liste članova također ažuriraju.
			
				\begin{figure}[H]
				\centerfloat
				\includegraphics[scale=0.25]{dijagrami/Dijagram aktivnosti - Revidiranje dojave o pogrešci u taktici.jpg} %veličina slike u odnosu na originalnu datoteku i pozicija slike
				\caption{Dijagram aktivnosti za revidiranje dojave o pogrešci na taktici}
				\label{fig:UC9}
			\end{figure}
			
			\eject
		\section{Dijagram komponenti}
		
			\textbf{\textit{dio 2. revizije}}\\
		
			 \textit{Potrebno je priložiti dijagram komponenti s pripadajućim opisom. Dijagram komponenti treba prikazivati strukturu cijele aplikacije.}