\chapter{Opis projektnog zadatka}
		
		\textbf{\textit{dio 1. revizije}}\\
		
		\textit{Na osnovi projektnog zadatka detaljno opisati korisničke zahtjeve. Što jasnije opisati cilj projektnog zadatka, razraditi problematiku zadatka, dodati nove aspekte problema i potencijalnih rješenja. Očekuje se minimalno 3, a poželjno 4-5 stranica opisa.	Teme koje treba dodatno razraditi u ovom poglavlju su:}
		\begin{packed_item}
			\item \textit{potencijalna korist ovog projekta}
			\item \textit{postojeća slična rješenja (istražiti i ukratko opisati razlike u odnosu na zadani zadatak). Dodajte slike koja predočavaju slična rješenja.}
			\item \textit{skup korisnika koji bi mogao biti zainteresiran za ostvareno rješenje.}
			\item \textit{mogućnost prilagodbe rješenja }
			\item \textit{opseg projektnog zadatka}
			\item \textit{moguće nadogradnje projektnog zadatka}
		\end{packed_item}
		
		\textit{Za pomoć pogledati reference navedene u poglavlju „Popis literature“, a po potrebi konzultirati sadržaj na internetu koji nudi dobre smjernice u tom pogledu.}
		
		\section{Motivacija za projekt}
		Ideja ovog je projekta olakšati šahovskom klubu obavljanje administrativnih poslova i komunikaciju s članovima. To bi se postiglo web aplikacijom sa funkcionalnostima kao što su online uplaćivanje članarine, objavljivanje novosti, te prijavljivanje članova za treninge i turnire. Web aplikacija bila bi intuitivna i jednostavna za korištenje kako članovima, tako i zaposlenicima kluba. Osim toga, članovi bi se putem web aplikacije mogli nadmetati u rješavanju dnevnih taktika, te tako sudjelovati u šahovskoj zajednici čak i van fizičkog prostora kluba.  \\
Još jedan bitan aspekt ove aplikacije bio bi njen potencijal za nadogradnju. Jasno je da će u budućnosti potrebe šahovskog kluba rasti i da će se trebati uvesti nove funkcionalnosti u web aplikaciju. Tu činjenicu mora odražavati kvaliteta programskog rješenja i dokumentacije kako bi se pružio dobar temelj za nova proširenja.
		
		\eject

		
	